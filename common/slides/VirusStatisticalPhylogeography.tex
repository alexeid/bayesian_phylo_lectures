%%%%%%%%%%%%%%%%%%%%
% STATISTICAL PHYLOGEOGRAPHY
%%%%%%%%%%%%%%%%%%%%

% INTRODUCTION TO STATISTICAL PHYLOGEOGRAPHY
\begin{frame}
\frametitle{Statistical phylogeography of infectious diseases}

What can the geographical distribution of genetically-typed infectious diseases tell us?

\begin{itemize}
\item Infectious diseases as genetic markers of their host?
\item Geographical origin of infectious disease outbreaks
\item Geographical and habitat features relevant to spread and colonization
\item Identification of major migration/transmission routes
\end{itemize}

\end{frame}

% INTRODUCTION TO STATISTICAL PHYLOGEOGRAPHY
\begin{frame}
\frametitle{Models for statistical phylogeography}

\begin{itemize}
\item Continuous-time Markov processes on discrete locations - e.g. Lemey {\it et al} (2009)
\item Structured Coalescent on discrete locations - e.g. Ewing {\it et al} (2004), Beerli and Felsenstein (2001)
\item Stochastic spatial epidemic models - later in this talk - Kuehnert, Drummond, {\it et al}
\item Independent Contrasts type models on continuous field - e.g. Biek {\it et al} (2006)
\item Inhomogeneous diffusions on high-resolution maps - e.g. Bouckaert, Drummond {\it et al} (in prep)
\end{itemize}

\end{frame}

\begin{frame}
\frametitle{Statistical phylogeography}

\begin{itemize}
\item Post-tree reconstruction analysis
\begin{itemize}
\item Discrete geographic locations
\item Less computationally intensive
\item Outcome of analysis depends on input tree
\item Migration route inferred by maximum parsimony (Maddison and Maddison, 2005; Wallace et al., 2007)
\item Can�t quantify uncertainty
\end{itemize}
\item Jointly estimates the phylogeny and phylogeographic parameters of interest.
\end{itemize}
\end{frame}

\begin{frame}
\frametitle{Mugration models}

\begin{itemize}
\item Mutation model used to analyse migration process (Lemey 2009)
\begin{itemize}
\item Geographic movement modelled by Markov process
\item Facilitates the migration rates between pairs of locations
\item Use BVS to infer dominant migration routes and average over uncertainty in the connectivity between different locations
\item Applied to find the origin and the path of the global spread of H5N1 and H1N1 2009
\item Denser sampling of locations provides improved phylogeographic estimates
\end{itemize}
\item Denser sampling increases the state space and hence computationally more intensive
\begin{itemize}
\item Solution: parallel computing on large numbers of GPU cores
\end{itemize}
\end{itemize}

\end{frame}

% SLIDE ON INFLUENZA A
\begin{frame}
\frametitle{Phylogeography of Influenza A in East Asian cities}
\framesubtitle{Stochastic variable selection of dominant migration routes}
\begin{centering}%
\includegraphics[width=\textwidth]{../images/phylogeography/influenza1}%
\par%
\end{centering}%
\end{frame}

\begin{frame}
\frametitle{The structured coalescent}

\begin{itemize}
\item Accommodates subdivision (demes) in the population
\item First described by Hudson (1990)
\item Available in Migrate (Beerli and Felsenstein, 2001)
\begin{itemize}
\item Estimates subpopulation sizes and migration rates in both ML and Bayesian framework
\end{itemize}
\item Extensions
\begin{itemize}
\item Serial sampling of data (Ewing et al., 2004)
\item Number demes change over time (Ewing and Rodrigo, 2006)
\item Ghost demes � demes that are hidden/not sampled (but you know they are there; Beerli, 2004; Ewing and Rodrigo, 2006b)
\end{itemize}
\end{itemize}

\end{frame}

\begin{frame}
\frametitle{Modeling phylogeography on a spatial continuum}

\begin{itemize}
\item Phylogeography of wildlife host population
\begin{itemize}
\item Modelled in spatial continuum by diffusion models
\item Tend to be poorly modelled by a small number of discrete demes
\item E.g. expansion of geographic range in eastern United States of the raccoon-specific rabies virus (Biek et al., 2007; Lemey et al., 2010).
\end{itemize}
\end{itemize}

\end{frame}

% SLIDE 1 ON RABIES IN EASTERN USA
\begin{frame}
\frametitle{Spatial Expansion of Epizootic Rabies in Raccoons}
\framesubtitle{Biek {\it et al} (2007) {\it PNAS}, {\bf 104}:7993-7998 }

\begin{centering}%
\includegraphics[width=0.6\textwidth]{../images/phylogeography/Biek2007_Figure3}%
\par%
\end{centering}%
\end{frame}

% SLIDE 2 ON RABIES IN EASTERN USA
\begin{frame}
\frametitle{Spatial Expansion of Epizootic Rabies in Raccoons}
\framesubtitle{Biek {\it et al} (2007) {\it PNAS}, {\bf 104}:7993-7998 }

\begin{centering}%
\includegraphics[width=0.9\textwidth]{../images/phylogeography/Biek2007_Figure2}%
\par%
\end{centering}%
\end{frame}

\begin{frame}
\frametitle{Model phylogeography by Brownian diffusion}

\begin{itemize}
\item Via comparative method (Felsenstein, 1985; Harvey and Pagel, 1991)
\item Example of Brownian diffusion application
\begin{itemize}
\item Applied to model the phylogeography of Feline Immunodeficiency Virus collected from cougar (Puma concolor) population around western Montana (Biek, 2006)
\item Virus is predominantly vertically transmitted
\item Therefore the phylogeographic results of the virus was used as a proxy of the host population structure and history.
\end{itemize}
\end{itemize}

\end{frame}

% SLIDE ON COUGARS
\begin{frame}
\frametitle{A Virus Reveals Population Dynamics of its Carnivore Host}
\framesubtitle{Biek {\it et al} (2006) {\it Science}, {\bf 311}:538-541}

\begin{centering}%
\includegraphics[width=\textwidth]{../images/phylogeography/311_538_F2}%
\par%
\end{centering}%
\end{frame}

\begin{frame}
\frametitle{Relaxed random walk}

\begin{itemize}
\item Brownian diffusion assumes rate of dispersal is the same across all branches
\begin{itemize}
\item Extend the concept of relaxed clock model (Drummond, 2006) to diffusion process (Lemey, 2010)
\item Has better coverage and statistical efficiency, when the underlying process of spatial movement resembles a over-dispersed random walk
\end{itemize}
\end{itemize}

\end{frame}

\begin{frame}[plain]

\begin{centering}

\includegraphics[height=\textheight]{../images/phylogeography/relaxed1}

\end{centering}

\end{frame}

\begin{frame}[plain]

\begin{columns}
\column{0.5\textwidth}

\includegraphics[width=\textwidth]{../images/phylogeography/relaxed2}

\column{0.5\textwidth}

\includegraphics[width=\textwidth]{../images/phylogeography/relaxed3}

\end{columns}

\end{frame}

\begin{frame}[plain]

\begin{centering}
\includegraphics[width=\textwidth]{../images/phylogeography/relaxed4}
\end{centering}

\end{frame}
