% UNCONSTRAINED VERSUS CLOCK-CONSTRAINED

\begin{frame}
\frametitle{Model assumptions}

\begin{columns}

\column{0.5\textwidth}

\begin{centering}

\includegraphics[height=0.33\textheight]{../images/relaxedClocks/unconstrainedTree}

\end{centering}

\column{0.5\textwidth}

\begin{centering}

\includegraphics[height=0.33\textheight]{../images/relaxedClocks/clockConstrainedTree}

\end{centering}

\end{columns}

\begin{columns}[t]

\column{0.5\textwidth}

\small{
\begin{itemize}
\item Product of rate and time (branch length) is independent and identically distributed among branches.
\item The root of the tree could be anywhere with equal probability.
\item Topology implies nothing about individual branch lengths.
\end{itemize}
}

\column{0.5\textwidth}

\small{
\begin{itemize}
\item Rate of evolution is the same on all branches.
\item The root of the tree is equidistant from all tips.
\item Topology constrains branch lengths (e.g. two branches in a cherry must be of equal length)
\end{itemize}
}
\end{columns}

\end{frame}